% Colours! 
\newcommand{\targetcolourmodel}{cmyk} % rgb for a digital version, cmyk for a printed version. Only use lowercase
\selectcolormodel{\targetcolourmodel}

% Define colours from https://medarbejdere.au.dk/en/administration/communication/guidelines/guidelinesforcolours/
\definecolor{aublue}	{cmyk}{1,0.8,0,0.15}
%\definecolor{blue}      {rgb/cmyk}{0.1843,0.2431,0.9176 / 0.88,0.76,0,0}
%\definecolor{brightgreen}{rgb/cmyk}{0.1216,0.8157,0.5098 / 0.69,0,0.66,0}
%\definecolor{navyblue}  {rgb/cmyk}{0.0118,0.0588,0.3098 / 1,0.9,0,0.6}
%\definecolor{yellow}    {rgb/cmyk}{0.9647,0.8157,0.3019 / 0.05,0.17,0.82,0}
%\definecolor{orange}    {rgb/cmyk}{0.9882,0.4627,0.2039 / 0,0.65,0.86,0}
%\definecolor{pink}      {rgb/cmyk}{0.9686,0.7333,0.6941 / 0,0.35,0.26,0}
%\definecolor{grey}      {rgb/cmyk}{0.8549,0.8549,0.8549 / 0,0,0,0.2}
%\definecolor{red}       {rgb/cmyk}{0.9098,0.2471,0.2824 / 0,0.86,0.65,0}
%\definecolor{green}     {rgb/cmyk}{0,0.5333,0.2078 / 0.89,0.05,1,0.17}
%\definecolor{purple}    {rgb/cmyk}{0.4745,0.1373,0.5569 / 0.67,0.96,0,0}

\newcommand{\frontpagetextcolour}{white} % front page text colour: white or black
\colorlet{frontbackcolor}{aublue} % Set the background colour of the front- and back page.
\newcommand{\frontfont}{\rm}

% Font

\renewcommand*\sfdefault{lmss}
\renewcommand*\ttdefault{txtt}

% Table of contents (TOC) and numbering of headings
\setcounter{tocdepth}{1}    % Depth of table of content: sub sections will not be included in table of contents
\setcounter{secnumdepth}{2} % Depth of section numbering: sub sub sections are not numbered

\makeatletter % Reset chapter numbering for each part
\@addtoreset{chapter}{part}
\makeatother

% Spacing of titles and captions
%\titlespacing\chapter{0pt}{0pt plus 4pt minus 2pt}{4pt plus 2pt minus 2pt}
%\titlespacing\section{0pt}{12pt plus 3pt minus 3pt}{2pt plus 1pt minus 1pt}
%\titlespacing\subsection{0pt}{8pt plus 2pt minus 2pt}{0pt plus 1pt minus 1pt}
%\titlespacing\subsubsection{0pt}{4pt plus 1pt minus 1pt}{-2pt plus 1pt minus 1pt}
%\captionsetup{belowskip=\parskip,aboveskip=4pt plus 1pt minus 1pt}

% Setup header and footer
\fancypagestyle{main}{% All normal pages
  \fancyhead{}
  \fancyfoot{}
  \renewcommand{\headrulewidth}{0pt}
  \fancyfoot[LE,RO]{\footnotesize \thepage}
  \fancyfoot[RE,LO]{\footnotesize \thesistitle} % - \rightmark
  \fancyhfoffset[E,O]{0pt}
}
\fancypagestyle{plain}{% Chapter pages
  \fancyhead{}
  \fancyfoot{}
  \renewcommand{\headrulewidth}{0pt}
  \fancyfoot[LE,RO]{\footnotesize \thepage}
  \fancyfoot[RE,LO]{\footnotesize \thesistitle} % - \leftmark
  \fancyhfoffset[E,O]{0pt}
}

% Hypersetup
\hypersetup{
  pdfauthor={\thesisauthor},
  pdftitle={\thesistitle},
  %pdfsubject={\thesissubtitle},
  pdfdisplaydoctitle,
  bookmarksnumbered=true,
  bookmarksopen,
  breaklinks,
  linktoc=all,
  plainpages=false,
  unicode=true,
  colorlinks=true,
  linkcolor=aublue,
  citecolor=aublue,
  urlcolor=aublue,
  %hidelinks,                        % Do not show boxes or coloured links.
}

\captionsetup[lstlisting]{font={scriptsize}}
\usepackage[scaled=.95]{inconsolata}

% Listings setup
\lstset{
  language=C++,
  basicstyle=\footnotesize\ttfamily,% the size of the fonts that are used for the code
  commentstyle=\color{green},	    % comment style
  keywordstyle=\bfseries\ttfamily\color{aublue}, % keyword style
  numberstyle=\sffamily\tiny\color{aublue!80}, % the style that is used for the line-numbers
  stringstyle=\color{purple},	    % string literal style
  rulecolor=\color{aublue},		% if not set, the frame-color may be changed on line-breaks within not-black text (e.g. comments (green here))
  breakatwhitespace=false,	      % sets if automatic breaks should only happen at whitespace
  breaklines=true,		      % sets automatic line breaking
  captionpos=b, 	      % sets the caption-position to bottom
  deletekeywords={},		      % if you want to delete keywords from the given language
  escapeinside={\%*}{*)},	      % if you want to add LaTeX within your code
  frame=l,			 % adds a frame around the code
  %frameshape={NYNNNN}{yn}{ny}{NYNNNN},
  framerule=2pt,
  framesep=5pt,
  rulesep=3pt,
  xleftmargin=7.5pt,
  xrightmargin=0pt,
  numbersep=13pt,		      % how far the line-numbers are from the code
  numbers=left, 	      % where to put the line-numbers; possible values are (none, left, right)
  showspaces=false,		      % show spaces everywhere adding particular underscores; it overrides 'showstringspaces'
  showstringspaces=false,	      % underline spaces within strings only
  showtabs=false,		      % show tabs within strings adding particular underscores
  stepnumber=1, 	      % the step between two line-numbers. If it's 1, each line will be numbered
  tabsize=2,			      % sets default tabsize to 2 spaces
  title=\lstname,		      % show the filename of files included with \lstinputlisting; also try caption instead of title
  floatplacement=htb,
  float,
}

\lstdefinelanguage{faust}{
  language=C++,
  keywords={component, declare, environment, import, library, process},
  emph={[2]ffunction, fconstant, fvariable},
  emph={[3]button, checkbox, vslider, hslider, nentry, vgroup, hgroup, tgroup, vbargraph, hbargraph, attach},
}

% Inline C++
\newcommand{\cpp}[1]{\lstinline[language=c++]{#1}}
\newcommand{\concept}[1]{\ensuremath{\textsc{#1}}}

% Signature field
\newlength{\myl}
\newcommand{\namesigdatehrule}[1]{\par\tikz \draw [black, densely dotted, very thick] (0.04,0) -- (#1,0);\par}
\newcommand{\namesigdate}[2][]{%
  \settowidth{\myl}{#2}
  \setlength{\myl}{\myl+10pt}
  \begin{minipage}{\myl}%
    \begin{center}
      #2	% Insert name from the command eg. \namesigdate{\authorname}
      \vspace{1.5cm} % Spacing between name and signature line 
      \namesigdatehrule{\myl}\smallskip % Signature line and a small skip
      \small \textit{Signature} % Text under the signature line "Signature"
      \vspace{1.0cm} % Spacing between "Signature" and the date line
      \namesigdatehrule{\myl}\smallskip % Date line and a small skip
      \small \textit{Date} % Text under date line "Date" 
    \end{center}
  \end{minipage}
}

% For the back page: cleartoleftpage
\newcommand*\cleartoleftpage{%
  \clearpage
  \ifodd\value{page}\hbox{}\newpage\fi
}

\newcommand{\todo}[1]{{\color[rgb]{.5,0,0}\textbf{$\blacktriangleright$#1$\blacktriangleleft$}}}

% Word/Page counts
\newread\tmpcount
\newcommand{\quickcharcount}[1]{%
  \immediate\write18{texcount -1 -sum -merge -char #1.tex > /tmp/chars.sum}%
  \openin\tmpcount=/tmp/chars.sum%
  \read\tmpcount to \thecharcount%
  \closein\tmpcount%
  \immediate\write18{rm /tmp/chars.sum}%
}

\newcommand{\quickwordcount}[1]{%
  \immediate\write18{texcount -1 -sum -merge #1.tex > /tmp/words.sum}%
  \openin\tmpcount=/tmp/words.sum%
  \read\tmpcount to \thewordcount%
  \closein\tmpcount%
  \immediate\write18{rm /tmp/words.sum}%
}

\newcommand{\quickpagecount}[1]{%
\quickcharcount{#1}%
\quickwordcount{#1}%
\thecharcount characters and approximately \thewordcount spaces =
{\xinttheiexpr[2](\thewordcount+\thecharcount)/4200\relax} standard pages%
}

\graphicspath{{Pictures/}}
