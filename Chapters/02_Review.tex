\chapter{Related Work}
\label{chap:related}

This chapter will briefly introduce FAUST, along with two examples of C++ libraries from other domains that
use techniques applied in this project to solve similar issues.

\section{FAUST}
\label{sec:related_faust}
As already mentioned in the introduction, this project builds largely on the research performed by the FAUST
authors. It was first developed Orlarey, Fober and Letz at Grame, Centre National de Creation Musciale, Lyon,
France in 2002\autocite{orlarey2002}, and has for the past 20 years been the basis of a lot of research in the
area of expressive DSP programming languages\autocite{letz:hal-02158925, buffa:hal-01721483, dimitrov:hal-02158954, gaster2018, berdahl:hal-03162897}.

The design of FAUST is based on block diagrams, and thus models a functional pipeline, abstracting away all
implementation details regarding buffers, sample rates etc, and instead represents the operations performed
on the signal, as opposed to the individual samples. This means the language is closer to the specification
of DSP algorithms, resulting in code that is both easier to write initially and to modify when changing the
specification.

FAUST programs can be compiled to various targets, including C++, JAVA, WebAssembly etc., and in addition to
targeting multiple languages, the user has a lot of control over the generated code, allowing for integration
with most frameworks. Going further, FAUST provides a lot of related tooling\autocite{faustweb}, such as
generating visualizations of the block diagram of a program, along with live-coding environments and
automatic UI generation.

Even with all of the tooling and integration options provided, FAUST still presents some issues. Firstly,
stateful components and algorithms must themselves be written in FAUST, which means that the vast amount of
existing, and often highly optimized C++ algorithms for DSP\autocite{FFTW05} cannot easily by used from
within FAUST, often resulting in FAUST programs eventually being rewritten in C++ when certain requirements
are met. Along the same vein, FAUST does not support multirate algorithms (see \autoref{chap:multirate}), which
makes an important set of programs impossible to implement efficiently.

During the following chapters, I will present a solution that attempts to address these issues, while
preserving the syntax and semantics of FAUST.

%%%%%%%%%%%%%%%%%%%%%%%%%%%%%%%%%%%%%
\section{Eigen}

Eigen is \emph{"a C++ template library for linear algebra: matrices, vectors, numerical solvers, and related algorithms"}\autocite{eigenweb}. It is often used as the canonical example of EDSLs in
C++, and is notable for being one of the oldest and most well-tested cases of this technology. It is
currently used by projects like Google's Tensorflow\autocite{tensorflow2015-whitepaper} and MIRA, a middleware for
robotics\autocite{Einhorn2012-bx}.

The Library provides, amongst other things, a simple and efficient interface for matrix operations, which are
implemented using operator overloading and expression templates. This means syntax that closely represents
the mathematical operators, but can be evaluated lazily and using optimizations based on the entire
expression. Thus, Eigen shows the viability of highly optimized EDSLs in C++.

Since Eigen was first released in 2006, C++ has gone through a lot of changes, but for backwards
compatibility reasons it still targets C++98. Especially in the metaprogramming areas of C++ a lot of new
features have been introduced to simplify and improve the programming experience, meaning the specific
internals of Eigen are not as applicable in a modern context.

%%%%%%%%%%%%%%%%%%%%%%%%%%%%%%%%%%%%%
\section{C++20 Ranges}

C++20\autocite{C++Std} merged the long anticipated ranges proposal\autocite{P0896R4}, which was
significant in a number of ways. First of all, it added a lot of fairly simple utility functions and quality
of life improvements to working with containers and algorithms, most notibly range-based versions of all
standard algorithms. This means that functions like \cpp{std::sort} can now be called with any \emph{range} (i.e. container-like ) as
its first and only argument, instead of the interface being based on separate begin and end iterators.

The most notable part of the C++ ranges library in this context however, is that it contains composable
\emph{views}, and was the first major part of the standard library to be designed around concepts
and constraints\autocite{N4674}. The views provided by the ranges library are a simple case of using
operator overloading to build expression trees embedded in the type system, and using these to provide
features like lazy evaluation transparently. Thus, parts of the C++20 Ranges library can be used as an
example of how to implement an EDSL in C++20.
