\chapter{Conclusion \& Further Work}

Building on existing research, this thesis has explored an alternative way to write DSP programs modelled as
block diagrams. I have built an embedded DSL in a C++ library by applying existing techniques from other
domains to a this domain, and used that library to write an audio plugin that can be used in audio editors.
Furthermore I have shown the advantages of working inside a general purpose language, and specifically how
constrained C++ templates can provide a framework where parts of the algorithms can be written in
conventional C++, using the proven constructs and structures of the language in low-latency situations. I
have also explored multirate DSP as an example of an easy extension to the library to solve a specific signal
processing problem that requires more specialized design to implement in a DSL compiler.

While the testing of the library performed so far is minimal, it gives an indication of what is possible
using the ideas presented. Even in its current unoptimized state, the library only has a performance overhead
of around 1.5x compared to handwritten code. While this number is based on only a single example, it is an
example that covers most areas of the library, and should give an indication of the magnitude. I have also
presented some ideas on how to improve the performance, such as an evaluator that uses vector instructions
and optimizing the AST before evaluation. In both of these areas a lot of prior research exists, that could
be applied to this project.

Further enhancements could also be made to the syntactical aspects provided by the library, such as naming
signals as variables, and declaring blocks with a function syntax. However, the current syntax has been shown
to be useful already, and provide a good mapping from blocks diagrams of signal processors to code.

Lastly, as described in \autoref{chap:multirate}, multirate DSP is a complex problem that can be highly
optimized. The current implementation shows the basics of how it could be implemented, but for real use it
should be improved with better filters, as well as expanded to support undersampling in addition to
oversampling.

All in all, I believe that this project has shown the idea of using an EDSL built on block algebra to be
viable, as well as providing the basis of how such a library could be implemented.

