\chapter{The C++ Library}
\label{chap:cpp_lib}

Using the block algebra introduced in the previous chapter, I have developed an embedded domain specific
language as a C++ Library called EDA (Expressive DSP for Audio). The goal of this library is to implement a
syntax based on the algebra of blocks, and thus resembling FAUST, while staying within the C++ ecosystem, and
allowing easy integration in both directions, i.e. using EDA within other C++ frameworks
\emph{and} using other C++ libraries and algorithms within EDA.

The library is based on expression templates\autocite{bachelet}, which is the idea of representing an AST
as a static tree of templated types, and then building this AST by overloading operators. For example, the
expression
\cpp{a + (b * 2)} would evaluate to an object of the type \cpp{Plus<Var, Mult<Var, Literal>>} or similar. This idea can
be used to use native expression syntax to build the AST of the expression instead of actually evaluating it.
A further introduction to expression templates in general, and how to implement them in C++ can be found in
Bachelet/Yon (2017)\autocite{bachelet}, but the following sections go through this specific
implementation.

It is worth noting that the library makes heavy usage of certain C++20\autocite{C++Std} features,
especially the features commonly refered to under the umbrella term
\emph{concepts}\autocite{cppr:concepts}, which help constrain the template parameters of templated
entities.

The code library code and all tests and examples can be found in \autoref{appendix}, and on
\url{https://github.com/topisani/eda}.

\section{Blocks}

One important design decision in this library, is to split the declaration of a block diagram from the
evaluation of the corresponding signal processor. While this makes implementing new block types slightly more
verbose, it has a couple of advantages. Most importantly, a block diagram is a static structure that can be
declared once, even constructed at compile time in many cases, and then multiple instances of the signal
processor can be constructed at runtime as needed. Secondly, having the block diagram available as a
declarative structure makes other evaluators than the signal processor possible, such as one that builds a
visualization of the block diagram.

\subsubsection{BlockBase}

As defined in the previous chapter, a block in $\mathbb D$ has a number of inputs and a number of
outputs. In EDA, these are modelled by extending the \cpp{BlockBase} CRTP\footnotemark{} base class, meaning a base class
template that is always passed the derived class as its first template parameter:

\footnotetext{Curriously Recurring Template Pattern, see \url{https://en.wikipedia.org/wiki/Curiously_recurring_template_pattern}}

\newpage
\begin{cppcodenl}
  template<typename Derived, std::size_t InChannels, std::size_t OutChannels>
  struct BlockBase {
    static constexpr std::size_t in_channels = InChannels;
    static constexpr std::size_t out_channels = OutChannels;

    constexpr auto operator()(auto&&... inputs) const noexcept
    requires(sizeof...(inputs) <= InChannels);
  };
\end{cppcodenl}

This base class provides the \cpp{in_channels} and \cpp{out_channels} constants, along with the call operator used
for partial application (see \autoref{sec:eda_partial_application}).

\subsubsection{AnyBlock, AnyBlockRef and ABlock}

Three basic concepts are introduced as well to check whether a type \cpp{T} is a block, a
block with or without reference/const/volatile qualifiers, or a block with a specific signature.
\concept{AnyBlock} also requires a type to model \cpp{std::copyable}\footnotemark, to make sure that
blocks can be copied around.

\footnotetext{See \url{https://en.cppreference.com/w/cpp/concepts/copyable}}

\begin{cppcodenl}
  template<typename T>
  concept AnyBlock = std::is_base_of_v<BlockBase<T, T::in_channels, T::out_channels>, T> && std::copyable<T>;

  template<typename T>
  concept AnyBlockRef = AnyBlock<std::remove_cvref_t<T>>;

  template<typename T, std::size_t I, std::size_t O>
  concept ABlock = AnyBlock<T> &&(T::in_channels == I) && (T::out_channels == O);
\end{cppcodenl}

\subsubsection{\cpp{ins<T>} and \cpp{outs<T>}}

To access the number of input/output channels, the following shorthand variable templates are introduced:

\begin{cppcodenl}
  template<AnyBlockRef T>
  constexpr auto ins = std::remove_cvref_t<T>::in_channels;

  template<AnyBlockRef T>
  constexpr auto outs = std::remove_cvref_t<T>::out_channels;
\end{cppcodenl}

They allow \cpp{ins<T> == 2} when \cpp{T} is a cv/ref-qualified block, i.e. it models \concept{AnyBlockRef}.

\subsection{Identity block}
As the simplest example of a block type declaration, I take a look at the identity block. For convenience, it
has here been extended with a template parameter \cpp{N} to allow for identity blocks of
different numbers of channels.
\cpp{ident<N>} is equal to the parallel composition of \cpp{N} identity blocks. As a side note, the \Cut block has been
extended in a similar manner.

\begin{cppcodenl}
  template<std::size_t N = 1>
  struct Ident : BlockBase<Ident<N>, N, N> {};

  template<std::size_t N = 1>
  constexpr Ident<N> ident;
\end{cppcodenl}

Here, the declaration consists of two parts, the \cpp{Ident} type itself, which inherits from \cpp{BlockBase}, and the constant
\cpp{ident} variable template, which serves the function of the constructor.

\section{Block Compositions}

\subsubsection{CompositionBase}

Block compositions are implemented as class templates that derive from \cpp{CompositionBase}, which iself
derives from
\cpp{BlockBase}. \cpp{CompositionBase} keeps a tuple of the operand blocks, which can then be accessed by
the deriving class. Likr \cpp{BlockBase}, it is a CRTP-style base class template, so its first
template parameter is the class that is deriving from it.

\begin{cppcodenl}
  template<typename D, std::size_t In, std::size_t Out, AnyBlock... Operands>
  struct CompositionBase : BlockBase<D, In, Out> {
    using operands_t = std::tuple<Operands...>;
    constexpr CompositionBase(Operands... ops) noexcept : operands(std::move(ops)...) {}
    operands_t operands;
  };
\end{cppcodenl}

\subsubsection{AComposition and ACompositionRef}
Once again, a couple of acompanying concepts are added to check that a type \cpp{T} is a
composition or reference to one:

\begin{cppcodenl}
  template<typename T>
  concept AComposition = AnyBlock<T> && requires (T& t) {
    typename T::operands_t;
    { t.operands } -> util::decays_to<typename T::operands_t>;
  };

  template<typename T>
  concept ACompositionRef = AComposition<std::remove_cvref_t<T>>;
\end{cppcodenl}

\subsubsection{\cpp{operands_t}}

As a shorthand for accessing the type of the operands of a cv-ref qualified composition, the
\cpp{operands_t<T>} alias template is added:

\begin{cppcodenl}
  template<ACompositionRef T>
  using operands_t = typename std::remove_cvref_t<T>::operands_t;
\end{cppcodenl}

\subsubsection{Sequential}

Recall the type rule for sequential composition from \autoref{sec:block_seq}:

\begin{prooftree}
  \AxiomC{$d_1 : n \rightarrow p$}
  \AxiomC{$d_2 : p \rightarrow m$}
  \BinaryInfC{$\Sequential(d_1, d_2) : n \rightarrow m$}
\end{prooftree}

Using type constraints, this can be encoded as the following block declaration:

\begin{cppcodenl}
  template<AnyBlock Lhs, AnyBlock Rhs>
  requires(outs<Lhs> == ins<Rhs>)
  struct Sequential : CompositionBase<Sequential<Lhs, Rhs>, ins<Lhs>, outs<Rhs>, Lhs, Rhs> {};
\end{cppcodenl}

First of all, \cpp{Lhs} and \cpp{Rhs} must both model the concept \cpp{AnyBlock}, which simply
ensures that the types given are in fact blocks. Secondly, a \emph{requires-clause} is added to the struct
declaration to assert that the outputs of \cpp{Lhs} is equal to the inputs of \cpp{Rhs}. If these requirements are unsatisfied,
the compiler emits useful error messages that are fairly easy to trace (see \autoref{sec:eda_errors}). The
second and third template parameters to \cpp{CompositionBase} specify the number of input and output
channels, so by passing \cpp{ins<Lhs>} and \cpp{outs<Rhs>} respectively,
\cpp{Sequential<Lhs, Rhs>} has the signature $n \rightarrow m$ as specified in the type rule. Finally, \cpp{Lhs} and
\cpp{Rhs} are passed as the \cpp{Operands...} argument to \cpp{CompositionBase}, meaning those blocks will be stored
in the \cpp{std::tuple<Lhs, Rhs> operands} member variable

For ease of construction, the free function \cpp{seq(a, b)} is written as follows:

\begin{cppcodenl}
  template<AnyBlockRef Lhs, AnyBlockRef Rhs>
  constexpr auto seq(Lhs&& lhs, Rhs&& rhs) noexcept
  {
    return Sequential<std::remove_cvref_t<Lhs>, std::remove_cvref_t<Rhs>>{
      .lhs = std::forward<Lhs>(lhs),
      .rhs = std::forward<Rhs>(rhs)
    };
  }
\end{cppcodenl}

\subsubsection{Remaining Binary compositions}

When declaring a block diagram (as opposed to when evaluating its signal processor), the only differences
between the various compositional operators are the requirements for the operands and the calculation of the
signature. For example, parallel composition is declared as follows:

\begin{cppcodenl}
  template<AnyBlock Lhs, AnyBlock Rhs>
  struct Parallel : CompositionBase<Parallel<Lhs, Rhs>, ins<Lhs> + ins<Rhs>, outs<Lhs> + outs<Rhs>, Lhs, Rhs> {};
\end{cppcodenl}

Here the signature is calculated differently from sequential composition, and there are no requirements on
\cpp{Lhs} and \cpp{Rhs}. Recursive, split, and merge composition are all
implemented similarly by simply translating the requirements and signature to C++ type requirements (see
appendix, page \pageref{code:remaining_blocks}, line \ref{code:remaining_blocks}).

\subsection{Arithmetic}

The arithmetic blocks defined in \autoref{sec:block_arithmetic} are declared as basic block types:

\begin{cppcodenl}
  struct Plus : BlockBase<Plus, 2, 1> {};
  struct Minus : BlockBase<Minus, 2, 1> {};
  struct Times : BlockBase<Times, 2, 1> {};
  struct Divide : BlockBase<Divide, 2, 1> {};
\end{cppcodenl}

And as with the \Ident and \Cut blocks, we declare constants to use the blocks:
\begin{cppcodenl}
  constexpr Plus plus;
  constexpr Minus minus;
  constexpr Times times;
  constexpr Divide divide;
\end{cppcodenl}

With these, we can start to build simple block diagrams, such as the following, which is drawn in
\autoref{fig:block_sum_and_diff}:

\begin{cppcodenl}
  ABlock<2, 2> auto d = split(par(plus, minus))
\end{cppcodenl}

\begin{figure}
  \centering
  \includestandalone[]{Pictures/block_sum_and_diff}
  \caption{A block diagram that computes the sum and difference of its inputs and outputs both}
  \label{fig:block_sum_and_diff}
\end{figure}

\section{Literals and References}
\label{sec:eda_lit_and_ref}

By now we know how to declare primitive and compositional blocks, so the following code should seem natural.
\newpage
\begin{cppcodenl}
  struct Literal : BlockBase<Literal, 0, 1> {
    float value;
  };

  constexpr Literal literal(float f) noexcept {
    return Literal{.value = f};
  }

  struct Ref : BlockBase<Ref, 0, 1> {
    float* ptr = nullptr;
  };

  constexpr Ref ref(float& f) noexcept
  {
    return Ref{.ptr = &f};
  }
\end{cppcodenl}

The \cpp{Literal} and \cpp{Ref} blocks each have a signature of $0 \rightarrow 1$, and can be used to
introduce scalar values into the block diagram. \cpp{Literal} is used for constants, and \cpp{Ref} is used for values that change
over time, i.e. non-signal values used to control parameters of the signal processor. This is a problem that
faust solves using functions that model UI elements, such as \cpp{vslider(name, ...)}\autocite{orlarey2004}

As a shorthand for \cpp{lit(0.5)}, the user-defined literal\footnote{see \url{https://en.cppreference.com/w/cpp/language/user_literal}} \cpp{0.5_eda} is also
provided:
\begin{cppcodenl}
  constexpr Literal operator"" _eda(long double f) noexcept
  {
    return literal(static_cast<float>(f));
  }
\end{cppcodenl}

\section{Operator overloading and shorthand syntax}

With the block types and construction functions, block diagrams can be declared by composing the
constructors. For example, the following declares a block diagram that calculates the relative difference
between two signals, i.e. given the values $a, b$, it outputs $\frac{a - b}{b}$:

\begin{cppcodenl}
  auto d = seq(par(ident<1>, split(ident<1>, ident<2>)), seq(par(minus, ident<1>), divide);
\end{cppcodenl}

This syntax quickly becomes very verbose, and the prefix functions make it hard to read. So, using operator
overloading and single-character constants, this section introduces syntax so the above can be rewritten as

\begin{cppcodenl}
  auto d = (_, _ << (_, _)) | ((_ - _) / _);
\end{cppcodenl}

\begin{figure}
  \centering
  \includestandalone[]{Pictures/block_reldiff}
  \caption{A block diagram that computes the relative difference between two signals}
  \label{fig:block_reldiff}
\end{figure}

\subsection{Selecting operators}

\newcommand{\oper}[1]{\EscVerb{#1}}

The first step is to decide which C++ operators will be mapped to which block diagram operations, and for
this the EDA library has to make some other choices than Faust, as the availability of overloadable operators
in C++ obviously places some constraints. The syntax here has been selected to stay as close as possible to
faust, but in practice one might consider some changes, as especially overriding the comma operator is
usually discouraged in idiomatic C++.

\begin{table}[]
  \centering
  \begin{tabular}{|l|c|c|}
    \hline
    $d$                     & Faust Syntax       & EDA Syntax         \\
    \hline\hline
    \Ident                  & \EscVerb{\_}       & \EscVerb{\_}       \\
    \Cut                    & \EscVerb{!}        & \EscVerb{\$}       \\
    $\Sequential(d_1, d_2)$ & \EscVerb{d1 : d2}  & \EscVerb{d1 | d2}  \\
    $\Parallel(d_1, d_2)$   & \EscVerb{d1 , d2}  & \EscVerb{d1 , d2}  \\
    $\Recursive(d_1, d_2)$  & \EscVerb{d1 ~ d2}  & \EscVerb{d1 \% d2} \\
    $\Split(d_1, d_2)$      & \EscVerb{d1 <: d2} & \EscVerb{d1 << d2} \\
    $\Merge(d_1, d_2)$      & \EscVerb{d1 :> d2} & \EscVerb{d1 >> d2} \\
    \hline
  \end{tabular}
  \caption{Faust and EDA syntax for basic block diagram components.}
  \label{tab:eda_block_syntax}
\end{table}

Firstly, for the \Ident and \Cut blocks, the variables \oper{\_} and \oper{\$} are
used:

\begin{cppcodenlimpl*}{escapeinside=||}
  constexpr Ident<1> _ = ident<1>;
  constexpr Cut<1> |\$| = cut<1>;
\end{cppcodenlimpl*}

For the compositional operators, most of them follow a similar structure. Sequential composition is done with
\oper{:} in faust, but since \oper{:} is not an operator in C++, the bitwise OR
operator (\oper{|}) is selected instead. This choice follows naturally from the C++20 ranges
library\autocite{C++Std}, which uses the operator for a similar purpose, and in turn is inspired by the
UNIX shell pipe\autocite{P0896R4}.

Parallel composition uses the \oper{,} operator, recursion has been changed to the modulo
operator  \oper{\%}, and split/merge use the left/right shift operators \oper{<<}
and \oper{>>}.

The final selection of operators and syntax compared to faust, can be seen in \autoref{tab:eda_block_syntax}. All in
all these operators follow a similar structure, so EDA and Faust programs should read fairly similarly.

\subsection{Operator overloads}

All of the selected operators can in C++ be overloaded as free functions, which makes the implementation very
simple. They all follow the exact same pattern, so as an example, here is the implementation of the
\oper{|} operator:

\begin{cppcodenl}
  template<typename Lhs, typename Rhs>
  constexpr auto operator|(Lhs&& lhs, Rhs&& rhs) noexcept
  requires(AnyBlockRef<Lhs> || AnyBlockRef<Rhs>)
  {
    return sequential(as_block(std::forward<Lhs>(lhs)), as_block(std::forward<Rhs>(rhs)));
  }
\end{cppcodenl}

Notably, this operator takes two references to arbitrary objects, and delegates to the constructor function
\cpp{sequential}. However, two things especially are of note.

Firstly, when implementing an operator overload template in namespace scope, it is very important to make
sure that the template arguments are properly constrained, since the operator would otherwise be valid in
ambiguous situations, such as when \cpp{Lhs} and \cpp{Rhs} are both \cpp{int}. The approach
used here is to use a requires-clause to make the operator only participate in overload resolution if either
\cpp{Lhs} \emph{or} \cpp{Rhs} to model \concept{AnyBlockRef}. Then, both arguments are passed through the
function
\cpp{as_block}, before being passed to \cpp{sequential}.

\subsubsection{Autoboxing of literals}

The purpose of this, is to allow literal floating-point values to be involved in EDA block expressions, i.e.
allow the expression \cpp{_ * 0.5}, which takes a signal and multiplies it by the floating point
value $0.5$. Without auto-boxing literals, the equivalent expression would be
\cpp{_ * literal(0.5)} (or \cpp{_ * 0.5_eda} with the user-defined literal introduced in \autoref{sec:eda_lit_and_ref}).

The implementation of \cpp{as_block} is very simple. It consists of two overloads, one that takes an
object that models \concept{AnyBlockRef}, and returns it unchanged, and one that takes a float, and wraps
it in a \cpp{Literal} block. By adding more \cpp{as_block} overloads, one can add implicit convertions from other types to
blocks as well, for example some may want a \cpp{as_block(float*)} overload to wrap a pointer to a float as a \cpp{Ref} block. This would allow expressions
of the form \cpp{(_ * &x)} to equal \cpp{(_ * ref(x))}.

\begin{cppcodenl}
  constexpr decltype(auto) as_block(AnyBlockRef auto&& input) noexcept
  {
    return std::forward<decltype(input)>(input);
  }

  constexpr Literal as_block(float f) noexcept
  {
    return literal(f);
  }
\end{cppcodenl}

As an extra utility, the \cpp{as_block_t<T>} type trait is defined to get the result type of calling \cpp{as_block} on an instance of \cpp{T}:

\begin{cppcodenl}
  template<typename T>
  using as_block_t = std::remove_cvref_t<decltype(as_block(std::declval<T>()))>;
\end{cppcodenl}

One problem with auto-boxing is that it still requires at least one operand to be a block type. This can
result in issues when for example \cpp{(_ , 2)} is a block that takes one signal, and outputs a
tuple of the input unchanged and the constant signal $2$, but
\cpp{(1, 2)} is just the value \cpp{int(2)}, and not a block with no inputs that outputs the constant signals
$1$ and $2$. This is because \cpp{(1, 2)} has selected the
standard C++ comma-operator, which returns its last argument, and not the overloaded comma-operator from EDA,
which builds a parallel block. This means the second expression would have to be written with at least one
argument explicitly converted to a block type, like \cpp{(1_eda, 2)}.  Whether this slight decrease in
verbosity is worth the added complexity is left up to the reader, as simply removing the
\cpp{as_block} calls and requiring both operands to model \concept{AnyBlockRef} should be an easy
change to make to the relevant code. For the sake of completeness, the library as described here includes
auto-boxing.

%%%%%%%%%%%%%%%%%%%%%%%%%%%%%%%%%%%%%%%%%%%%%%%%%%%%%%%%%%%%%%%%%%%%%%%%%%%%%%%%%%

\section{Evaluating Signal Processors}
\label{sec:eda_eval}

Up until now, this chapter has only described the declarations of block diagrams, and nothing about how to
evaluate the corresponding signal processors. As mentioned in the beginning of this chapter, the declarations
of block types are completely decoupled from the evaluators of the signal processors. This visitor-based
design makes sense for a couple of reasons, but most importantly, it allows for other visitors than just the
signal processor evaluator. For example, one could build a visitor that generates a visual representation of
the block diagram, or one that generates a user interface. Alternatively, other evaluators could be built,
for example one that vectorizes the operations (see Scaringella et al. \emph{Automatic vectorization in Faust} (2003)
\autocite{faust_auto_vec} for an approach that could be taken in EDA as well)

As presented here however, EDA contains only one visitor, the \cpp{evaluator<Block>}, which evaluates the
signal processor of a block for a single frame of data at a time, i.e. the values $(s_0(t), \dots, s_n(t))$ of a
signal tuple at a single time $t$. In code, this frame is represented as an instance of
the \cpp{Frame<N>} class, where
\cpp{N} is the number of channels. This class can mostly just be regarded as a wrapper around \cpp{std::array<float,
N>}, with some minor tweaks in construction. It also provides the free functions
\cpp{slice<I, J>(Frame<N>) -> Frame<J - I>} and \cpp{concat(Frame<N>, Frame<M>) -> Frame<N + M>}, which are mostly self explanatory, and
thus also omitted here. The full class and function definitions can be seen in the appendix, page
\pageref{codefile:include/eda/frame.hpp}.

To illustrate the relationship between the block declarations and evaluator instances, consider the following
program, which evaluates the difference between its current input and its previous input:

\begin{minted}[mathescape]{cpp}
  constexpr auto d = _ << (_ - mem<1>);
  auto a = make_evaluator(d);
  auto b = make_evaluator(d);
  
  a.eval(1) // $\Rightarrow 1 - 0 = 1$
  a.eval(2) // $\Rightarrow 2 - 1 = 1$
  b.eval(10) // $\Rightarrow 10 - 0 = 10$
  a.eval(5) // $\Rightarrow 5 - 2 = 3$
  b.eval(1) // $\Rightarrow 1 - 10 = -9$
\end{minted}

Notice the \cpp{constexpr} on line 1 - the block diagram is declared as a compile-time constant,
however, the evaluators  are runtime mutable, as they store the state of the signal processor. Continuing the
example, note that the two evaluators have separate memory, and can be executed independently from each
other. This is the basic usage of evaluators, and should give an understanding of the difference between
blocks as declarations of programs, and evaluators as instances of them.

\subsection{evaluator}
The basic evaluator class template is declared as such:

\begin{cppcodenl}
  template<AnyBlock T>
  struct evaluator;
\end{cppcodenl}

It is acompanied by the following type trait to extract the block type:

\begin{cppcodenl}
  template<typename T>
  struct block_for;

  template<typename T>
  struct block_for<evaluator<T>> {
    using type = T;
  };

  template<typename T>
  using block_for_t = typename block_for<T>::type;
\end{cppcodenl}

For each block type \cpp{T}, the class template \cpp{evaluator<T>} shall be specialized to define the evaluator. This
specialization shall model the \concept{AnEvaluator} concept, which is given in code below, and has the
following requirements:

\begin{enumerate}
  \item \label{item:derives} \cpp{T} shall publicly derive from \cpp{EvaluatorBase<block_for_t>} (see the following
        section).
  \item \cpp{T} shall be constructible from a const reference to an object of type \cpp{block_for_t<T>}.
  \item \cpp{T} shall have a member function \cpp{eval}, that takes a \cpp{Frame} of the appropriate number of channels,
        and returns a \cpp{Frame} of the appropriate number of channels, according to the signature of the
        block \cpp{block_for_t<T>}.
\end{enumerate}

\begin{cppcodenl}
  template<typename T>
  concept AnEvaluator = 
    std::derived_from<T, EvaluatorBase<block_for_t<T>>>
    && std::is_constructible_v<T, block_for_t<T> const&>
    && requires (T t, Frame<ins<block_for_t<T>>> in) {
      { t.eval(in) } -> std::convertible_to<Frame<outs<block_for_t<T>>>>;
    };
\end{cppcodenl}

To construct the evaluator of a block diagram, the factory function \cpp{make_evaluator} is supplied:

\newpage
\begin{cppcodenl}
  template<AnyBlockRef T>
  constexpr auto make_evaluator(T&& b)
  requires AnEvaluator<evaluator<std::remove_cvref_t<T>>>
  {
    return evaluator<std::remove_cvref_t<T>>(b);
  }
\end{cppcodenl}

\subsubsection{EvaluatorBase}

For primitive blocks, \cpp{EvaluatorBase} is an empty base class.

\begin{cppcodenl}
  template<AnyBlock T>
  struct EvaluatorBase {};
\end{cppcodenl}

It could however be used to add common functions to all evaluators, like a wrapper to \cpp{eval}
that works on whole buffers of frames. Other than that, its main purpose is for composition evaluators.

\subsubsection{Primitive block evaluator}
For primitive blocks, the evaluator implementations are fairly simple, one just needs to make sure to follow
the three requirements of \concept{AnEvaluator}, i.e. deriving from
\cpp{EvaluatorBase<Block>}, having a \cpp{evaluator(Block)} constructor, and implementing the proper \cpp{eval}
function.

As an example, here is the \cpp{evaluator<Plus>} specialization:

\begin{cppcodenl}
  template<>
  struct evaluator<Plus> : EvaluatorBase<Plus> {
    constexpr evaluator(Plus) {};
    Frame<1> eval(Frame<2> in)
    {
      return {in[0] + in[1]};
    }
  };
\end{cppcodenl}

\subsection{Composition Evaluators}

Evaluators of block compositions follow the exact same model, however, they need to defer to evaluators of
the operands, stored as member variables. This is done through the \cpp{EvaluatorBase} specialization for
blocks that model \concept{AComposition}, and using some light metaprogramming, the evaluators are stored in
a \cpp{std::tuple<evaluator<Operand>...>}, constructed from the operand blocks accessed through the block composition passed
to the constructor. The implementation looks like this:

\begin{cppcodenl}
  template<typename T>
  struct add_evaluator {};
    
  template<typename... Ts>
  struct add_evaluator<std::tuple<Ts...>> {
    using type = std::tuple<evaluator<Ts>...>;
  };
  
  template<typename T>
  using add_evaluator_t = typename add_evaluator<T>::type;

  template<AComposition T>
  struct EvaluatorBase<T> {
    constexpr EvaluatorBase(const T& t) : operands(t.operands) {}
    add_evaluator_t<operands_t<T>> operands;
  };
\end{cppcodenl}

The important part is that classes that derive from \cpp{EvaluatorBase<T>} can access the operand evaluators
through \cpp{std::get<I>(this->operands)}, where \cpp{I} is the index of the operands.

As in the other sections on compositions, the \Sequential composition is used as an example of an evaluator:

\begin{cppcodenl}
  template<AnyBlock Lhs, AnyBlock Rhs>
  struct evaluator<Sequential<Lhs, Rhs>> : EvaluatorBase<Sequential<Lhs, Rhs>> {
    constexpr evaluator(const Sequential<Lhs, Rhs>& block) : EvaluatorBase<Sequential<Lhs, Rhs>>(block) {}

    constexpr Frame<outs<Sequential<Lhs, Rhs>>> eval(Frame<ins<Sequential<Lhs, Rhs>>> in)
    {
      auto l = std::get<0>(this->operands).eval(in);
      return std::get<1>(this->operands).eval(l);
    }
  };
\end{cppcodenl}

Notice that this is all very similar to the primitive block evaluator shown earlier, and the only difference
is forwarding the block instance to the \cpp{EvaluatorBase} constructor.  The \cpp{eval} function implements the signal processor
for $\Sequential(l, r)$, by first directly calling the evaluator of the first operand, and passing the
result of that operation to the evaluator of the second operand.

The other compositional blocks are slightly more complicated to implement, but at this point it is just
conventional C++ in the \cpp{eval} functions. The full implementations can be seen in the
appendix, page \pageref{code:comp_eval}.

\section{Extra features}

One of the advantages of working inside C++ instead of in a DSL, is the ability to easily add features and
integrations with other parts of the C++ ecosystem. At the time of writing, the EDA library contains a few
such examples, and a couple of the more interesting ones are covered in this section. I will not go deep into
the implementation details here, but the code can be seen in the appendix, pages \pageref{code:extra_block} and
\pageref{code:extra_eval}.

\subsection{Functions}

A simple, but important enhancement to the library, is the \cpp{fun<I, O>(Callable)} block. It can be used to
easily wrap normal C++ functions to stateless blocks. As an example, the following defines blocks for the
hyperbolic tangent function (as used in \autoref{chap:multirate}), and floating point modulo.

\begin{cppcodenl}
  auto tanh = fun<1, 1>(&std::tanhf);
  auto mod = fun<2, 1>([](auto in) { return std::fmod(in[0], in[1]); });
\end{cppcodenl}

The \cpp{fun} block simply captures the function, and the evaluator delegates to it directly.
The implementation can be seen in the appendix, page \pageref{code:extra_block}.

A stateful version of \cpp{fun} also exists, which can store arbitrary state in the evaluator. \cpp{fun<I, O>(f, state_inits...)} copies \cpp{state_inits...}
into new objects on each evaluator construction, and the evaluator passes references to these objects as
extra arguments to \cpp{f(data, states...)}.

As an example, the following $\Block{counter} : 0 \rightarrow 1$ block outputs the signal $s(t) = t$, by
keeping a state of type \cpp{int}, initialized to $0$, and incrementing it
each time the evaluator is called:

\begin{cppcodenl}
  auto counter = fun<0, 1>([](Frame<0> in, int& state) { state++; return {state}}, 0);
\end{cppcodenl}

By using these constructs, most simple blocks, stateful or stateless, can be implemented without having to
write out the type and evaluator declarations.

\subsection{Function call syntax for blocks}
\label{sec:eda_partial_application}

FAUST has the ability to call blocks as functions, meaning $d(x_1, \dots, x_{\Ins(d)})$ is equal to
$\Sequential(\Parallel(x_1, \dots, x_{\Ins(d)}), d)$. To make this even more useful, partial application is supported, i.e.
% 
\begin{align*}
  d(x_1, \dots, x_n)      & ={} \Sequential(\Parallel(x_1, \dots, x_n, \Ident^{\Ins(d) - \sum^n_{i=1} \Outs(x_i)}), d) \\
  \sum^n_{i=1} \Outs(x_i) & \leq \Ins(d)
\end{align*}

This description can be translated directly to an implementation of the function call operator on
\cpp{BlockBase}, making it available to all blocks:

\begin{cppcodenl}
  template<typename D, std::size_t I, std::size_t O>
  constexpr auto BlockBase<D, I, O>::operator()(auto&&... inputs) const noexcept
  requires((outs<Inputs> + ...) <= I)
  {
    return seq(par(inputs..., ident<I - (outs<Inputs> + ...)>), *this);
  }
\end{cppcodenl}

\section{Compilation Errors}
\label{sec:eda_errors}

A big advantage of using C++20 requirements and constraints, is the vastly improved error context. As an
example, the following EDA expression violates the constraints of the \Sequential block.

\begin{cppcodenl}
auto block = (_, _) | (_);
\end{cppcodenl}

When compiled with clang v12.0.0, this code gives the following error message:

\begin{minted}{text}
In file included from tests/block.cpp:1:
include/eda/block.hpp:185:12: error: constraints not satisfied for class template 'Sequential' [with Lhs = eda::Parallel<eda::Ident<1>, eda::Ident<1>>, Rhs = eda::Ident<1>]
    return Sequential<std::remove_cvref_t<Lhs>, std::remove_cvref_t<Rhs>>{{FWD(lhs), FWD(rhs)}};
           ^~~~~~~~~~~~~~~~~~~~~~~~~~~~~~~~~~~~~~~~~~~~~~~~~~~~~~~~~~~~~~
include/eda/syntax.hpp:38:12: note: in instantiation of function template specialization 'eda::seq<eda::Parallel<eda::Ident<1>, eda::Ident<1>>, const eda::Ident<1> &>' requested here
    return seq(as_block(FWD(lhs)), as_block(FWD(rhs)));
           ^
tests/block.cpp:19:21: note: in instantiation of function template specialization 'eda::syntax::operator|<eda::Parallel<eda::Ident<1>, eda::Ident<1>>, const eda::Ident<1> &>' requested here
auto block = (_, _) | (_);
                    ^
include/eda/block.hpp:178:12: note: because 'outs<eda::Parallel<eda::Ident<1>, eda::Ident<1> > > == ins<eda::Ident<1> >' (2 == 1) evaluated to false
  requires(outs<Lhs> == ins<Rhs>) //
           ^
\end{minted}

Error messages emitted from programs that rely heavily on templates, but are implemented before or without
explicit constraints have notoriously long and cryptic error messages\footnotemark, so comparatively this
goes straight to the point.

\footnotetext{In his CppCon 2019 talk \emph{How to Implement Your First Compiler Feature:The Story of Concepts in Clang}\autocite{Raz2019}, Saar Raz famously talks about how he had to first implement a parser for the gigabyte of compiler error
  messages produced by one such program.}

To give the compiler as much context as possible for errors, it can be useful to use the
\concept{ABlock} concept as a placeholder type when declaring blocks:

\begin{cppcodenl}
  ABlock<1, 2> auto d = (_ << (_ + _, _ * _));
\end{cppcodenl}

Here the \cpp{ABlock<1, 2>} prefix becomes a checked annotation that \cpp{d} is a block with signature $1 \rightarrow 2$.

\section{Conclusion \& Further Work}
\label{sec:eda_conc}

In this chapter I have shown how an EDSL based on FAUSTs algebra of blocks can be implemented in C++, using
the basic techniques of expression templates to build an AST that can then be evaluated by a separate class.

The evaluator shown here is the simplest one possible, as it merely walks the AST, evaluating one sample at a
time. However, other evaluators could be explored that work more efficiently, such as a vectorizing evaluator
as mentioned in \autoref{sec:eda_eval}. Additionally, the AST represented by block diagram declarations
could be transformed and optimized, detecting patterns and redundancies that can be reduced to other blocks.

Additionally, visitors for other purposes could be built, for example one that generated a visual
representation of the block diagram, or more specialized ones that could be used to generate user interfaces
from block diagrams that contain special control blocks.

However, as presented here, the library can already be used to implement real applications, as will be
explored in the next chapter.

